%% LyX 2.2.3 created this file.  For more info, see http://www.lyx.org/.
%% Do not edit unless you really know what you are doing.
\documentclass[english]{article}
\usepackage[T1]{fontenc}
\usepackage[utf8]{luainputenc}
\usepackage{geometry}
\geometry{verbose,tmargin=2cm,bmargin=2cm,lmargin=2cm,rmargin=2cm,headheight=2cm,headsep=2cm}
\setlength{\parindent}{0bp}
\usepackage{float}
\usepackage{amssymb}
\usepackage{graphicx}

\makeatletter

%%%%%%%%%%%%%%%%%%%%%%%%%%%%%% LyX specific LaTeX commands.
%% Because html converters don't know tabularnewline
\providecommand{\tabularnewline}{\\}

\makeatother

\usepackage{babel}
\begin{document}

\section{Comportamiento de Amplificador Operacional Inversor}

A lo largo de esta seccion se procedera a analizar el comportamiento
ideal y real del amplificador operacional \emph{LM324 }conectado como
se muestra en la figura \ref{1_a_1}. Considerando los valores de
los componentes como se puede ver en la tabla \ref{1_a_t_1}.

\begin{figure}[H]
\begin{centering}
\includegraphics[scale=0.75]{Resources1a/Circuito1a}
\par\end{centering}
\caption{Circuito a analizar}
\label{1_a_1}
\end{figure}

\begin{table}[H]
\begin{centering}
\begin{tabular}{|c|c|c|c|}
\hline 
Caso & $R_{1}=R_{3}$ & $R_{2}$ & $R_{4}$\tabularnewline
\hline 
\hline 
1 & $10\left(k\Omega\right)$ & $100\left(k\Omega\right)$ & $40\left(k\Omega\right)$\tabularnewline
\hline 
2 & $10\left(k\Omega\right)$ & $10\left(k\Omega\right)$ & $40\left(k\Omega\right)$\tabularnewline
\hline 
3 & $100\left(k\Omega\right)$ & $10\left(k\Omega\right)$ & $400\left(k\Omega\right)$\tabularnewline
\hline 
\end{tabular}
\par\end{centering}
\caption{Valores de los componentes}
\label{1_a_t_1}

\end{table}

\subsection{Transferencia\label{subsec:1_a_1}}

Comenzando por el analisis ideal, se pidió calcular y graficar la
relación $\frac{V_{out}}{V_{in}}$, esto quiere decir, considerando
$a_{0}$ finito y $A(\omega)$ con polo dominante. Considerando las
siguientes ecuaciones descriptas a continuacion y operando correctamente,
se llega a que la relacion $\frac{V_{out}}{V_{in}}$ esta dada por
la ecuación (\ref{eq:1_a_1}).

\[
\left\{ \begin{array}{c}
V_{out}=-A(\omega)v^{-}\\
I=i_{3}+i_{1}\\
i_{1}=-i_{2}\\
v^{-}=i_{3}R_{3}\\
V_{in}-IR_{1}=v^{-}
\end{array}\right.
\]

\begin{equation}
H(s)=\frac{V_{out}}{V_{int}}=\frac{\frac{a_{0}R_{3}R_{2}}{R_{1}R_{2}+2R_{3}R_{1}-R_{3}R_{2}}}{1+\frac{s}{\omega_{p}\left(\frac{R_{1}R_{2}+2R_{3}R_{1}-R_{3}R_{2}}{R_{1}R_{2}+R_{3}R_{1}-R_{3}R_{2}}\right)}}\label{eq:1_a_1}
\end{equation}

Como podemos ver, tenemos un polo en nuestra transferencia por lo
cual, el circuito se deberia comportar a grandes rasgos como un pasabajos.
Es importante notar, que el valor de $R_{4}$ no afecta a la transferencia
del circuito. Si graficamos la transferencia de el circuito para los
distintos casos, podemos ver que, en efecto, se comporta como un pasabajos,
con diferente frecuencia de corte $f_{0}$, esto se puede ver en la
figura \ref{1_a_2}.

\begin{figure}[H]
\begin{centering}
\includegraphics[scale=0.5]{\string"Resources1a/transferencias a(w)\string".pdf}
\par\end{centering}
\caption{Comportamiento del circuito para los diferentes valores de impedancias}
\label{1_a_2}CHEQUEARR SI STO ESTA BIENN
\end{figure}

\subsection{Impedancia de entrada\label{subsec:1_a_2}}

Consecuentemente, se nos instó a calcular la impedancia de entrada
vista por el generador hacia nuestro circuito. Nuevamente, utilizando
las ecuaciones descriptas en la previa subseccion, y operando adecuadamente,
llegamos a que la impedancia de entrada es la descripta en la ecuación
(\ref{eq:1_a_2}).

\[
K=\frac{R_{2}a_{0}\omega_{p}(R_{3}+R_{1})-\omega_{p}(a_{0}-1)\left(R_{2}R_{3}+R_{1}R_{2}+R_{1}R_{3}\right)}{R_{2}a_{0}\omega_{p}-\left(R_{2}+R_{3}\right)\omega_{p}\left(a_{0}-1\right)}
\]

\[
C=\frac{\omega_{p}(a_{0}-1)\left(R_{2}R_{3}+R_{1}R_{2}+R_{1}R_{3}\right)-R_{2}a_{0}\omega_{p}\left(R_{3}+R_{1}\right)}{\left(R_{2}R_{3}+R_{1}R_{2}+R_{1}R_{3}\right)}
\]

\[
L=\frac{\left(R_{2}+R_{3}\right)\omega_{p}\left(a_{0}-1\right)-R_{2}a_{0}\omega_{p}}{R_{2}+R_{3}}
\]

\begin{equation}
\Rightarrow Z_{in}=K\,\frac{1+\frac{s}{C}}{1+\frac{s}{L}}\label{eq:1_a_2}
\end{equation}

Por lo tanto, para cada caso tendremos una impedancia de entrada como
se muestra en las siguientes formulas:

\[
Z_{in}=\frac{131946.78if+999790000}{6.911498if+99989}\,Caso\,1
\]

\[
Z_{in}=\frac{9424.77if+499985000}{0.628318if+49999}\,Caso\,2
\]

\[
Z_{in}=\frac{753981.6if+99998800000}{6.911498if+999989}\,Caso\,3
\]

Graficando la impedancia de entrada con respecto a la frecuencia de
entrada, se puede ver en la figura \ref{1_a_3}, como va variando
dependiendo de la frecuencia, es decir, no permanece constante. Nuevamente,
podemos observar como esta impedancia no es afectada por $R_{4}$.

\begin{figure}[H]
\begin{centering}
\includegraphics[scale=0.65]{Resources1a/zinp}
\par\end{centering}
\caption{Impedancia de entrada}
\label{1_a_3}

\end{figure}

\subsection{Consideraciones para utilizar un modelo lineal del OpAmp}

A continuación, se decidió aclarar cuales son las consideraciones
para caracterizar a nuestro circuito de manera lineal. Para esto poseemos
varias consideraciones que son descriptas a continuación.

\subsubsection{Saturación y polo dominante}

Si tenemos en cuenta un OpAmp ideal, nuestro primer contacto con un
circuito alineal se da cuando se entra en saturación, es decir, $\left|V_{out}\right|>\left|V_{cc}\right|$.
Si consideramos una tension de entrada de la forma $V_{in}=sin(2\pi ft)$,
es decir, con amplitud \emph{1(V)} \emph{, }solo nos basta con analizar
el valor del modulo de la transferencia vista en la ecuación (\ref{eq:1_a_1}).

\[
\left|H(f)\right|=\frac{\left|{R_{2}}\right|\left|{R_{3}}\right|\left|{a_{0}}\right|\left|\omega_{p}\right|\left|{R_{1}R_{2}+R_{1}R_{3}-R_{2}R_{3}}\right|}{\sqrt{4\pi^{2}f^{2}\left(R_{1}R_{2}+2R_{1}R_{3}-R_{2}R_{3}\right)^{2}+\omega_{p}^{2}\left(R_{1}R_{2}+R_{1}R_{3}-R_{2}R_{3}\right)^{2}}\left|{R_{1}R_{2}+2R_{1}R_{3}-R_{2}R_{3}}\right|}\leq V_{cc}
\]

\[
K=R_{1}R_{2}+2R_{1}R_{3}-R_{2}R_{3}
\]

\[
L=R_{1}R_{2}+R_{1}R_{3}-R_{2}R_{3}
\]

\[
f\geq\frac{\sqrt{-\left(V_{cc}\omega_{p}L\left|K\right|-\left|{R_{2}}\right|\left|{R_{3}}\right|\left|{a_{0}}\right|\left|{w}\right|\left|L\right|\right)\left(V_{cc}\omega_{p}\left(L\right)\left|K\right|+\left|{R_{2}}\right|\left|{R_{3}}\right|\left|{a_{0}}\right|\left|{w}\right|\left|L\right|\right)}}{2\pi\left(K\right)\left|{V_{cc}}\right|\left|K\right|}
\]

\[
\therefore\,f\geq26.5(kHz)\,Caso\,1
\]

\[
f\geq2.6(kHz)\,Caso\,2
\]

\[
f\geq265(Hz)\,Caso\,3
\]

\subsubsection{Slew Rate}

Otro problema con el cual nos topamos a la hora te poner limites a
nuestro circuito será el Slew Rate (SR), que indica el valor máximo
que puede tener $\frac{\partial V_{out}}{\partial t}$. Esto significa
que a una entrada \emph{x(t) }sinusoidal de la forma $x(t)=V_{p}sin(2\pi ft)$
le corresponde una salida $v_{out}(t)=\left|A(\omega)\right|V_{p}sin(2\pi ft+\phi(\omega))$
siendo $A(\omega)=\left|A(\omega)\right|e^{i\phi(\omega)}$. Por lo
tanto, derivando la salida nos queda la ecuación (\ref{eq:1_a_3}).

\begin{equation}
\frac{\partial v_{out}}{\partial t}=\left|A(\omega)\right|V_{p}2\pi f\,cos\left(2\pi ft+\phi(\omega)\right)\label{eq:1_a_3}
\end{equation}

A su vez, sabemos que, siempre y cuando $\phi(\omega)=0$ esa expresión
será máxima cuando \emph{t=0}. Por lo tanto;

\[
\frac{\partial v_{out}}{\partial t}\mid_{t=0}=\left|A(\omega)\right|V_{p}2\pi f\leq SR
\]

Por ende aproximando que $\left|A(\omega)\right|=a_{0}$ obtenemos
que la frecuencia de trabajo de nuestro circuito debe cumplir la ecuación
(\ref{eq:1_a_4}).

\begin{equation}
f\leq\frac{SR}{a_{0}2\pi V_{p}}\label{eq:1_a_4}
\end{equation}

Para el caso donde la tensión de entrada tenga un valor pico de 1(V),
$a_{0}=100000$ y $SR=...$ nos queda que para cada caso se deben
cumplir las siguientes ecuaciones.

\[
f\leq...\,Caso\,1
\]

\[
f\leq...\,Caso\,2
\]

\[
f\leq...\,Caso\,3
\]

\subsection{Aplicaciones y caracerísticas}

Como pudimos observar anteriormente, nuestro circuito es un pasabajos
inversor con un rango de frecuencias determinadas para cada caso,
durante esta sección nos centraremos en explicar algunas características
de nuestro circuito.

\subsubsection{Efecto de la resistencia R4 en el circuito inversor}

Como pudimos ver en las las subsecciones \ref{subsec:1_a_1} y \ref{subsec:1_a_2},
la transferencia y la impedancia de entrada no dependen del valor
de $R_{4}$, lo cual nos hace pregutarnos cual es el propósito de
esta resistencia. En principio, la resistencia tiene el objetivo de
cargar nuestro circuito para que funcione adecuadamente, esto querría
decir que la resistencia $R_{4}$ podría tomar cualquier valor entre
$0\,e\,\infty$, sin embargo, nuestro circuito presenta una corriente
de salida máxima y si hacemos tender $R_{4}\longrightarrow0$, la
corriente necesaria tendería a infinito, lo cual no es posible. El
otro caso posible es que $R_{4}\longrightarrow\infty$, esto significaria
que la corriente de salida del OpAmp sea la mínima, y es necesario
verificar que esa corriente no sea menor a la corriente minima de
salida del OpAmp. Sin embargo, como el segundo caso no suele traer
problemas, nos enfocaremos en procurar que la corriente de salida
no supere la corriente máxima nominal del amplificador operacional.
Para esto, y aproximando $i_{2}\approx0$, podemos decir que $R_{4}>\frac{V_{out}}{i_{max}}$.

\subsubsection{Efecto de la resistencia R3}

Por otro lado, podemos ver como, en la figura \ref{1_a_1}, la resistencia
$R_{3}$nos determina la tensión $v^{-}$. Sabiendo que $v^{+}=0(V)$,
significa que en cierta medida, la ganancia de nuestro circuito va
a estar dada por el valor de $R_{3}$ y en particular , si $R_{3}\longrightarrow0$,
entonces $v^{-}=0(v)$, por lo tanto $V_{out}=A(\omega)\,\left(v^{+}-v^{-}\right)=0(v)$,
con lo cual nuestra ganancia sería nula. De la misma manera, podemos
ver que si $R_{3}\longrightarrow\infty$, entonces la ganancia es
máxima.
\end{document}
