\usepackage[utf8]{inputenc}	%Necesitamos los 8 bits, por eso se usa utf8 para el ASCII
\usepackage[spanish, es-tabla, es-nodecimaldot]{babel} %Por defecto las tablas se llaman cuadro en babel spanish, es-tabla es para definir el cuadro como tabla
\usepackage{sansmathfonts}				% Sans Serif equations		
%Las ecuaciones las hagan con esta font, la biblioteca jodió con que se use uno y este es el más parecido
\renewcommand*\familydefault{\sfdefault} 		% Sans Serif as default font 
%\textsf{Lo que escribo acá se escribe en Sans Serif si no fuera el formato por defecto}
\usepackage[a4paper, 					% Page Layout
                     %showframe,				% This shows the frame
                     includehead,
                     footskip=7mm, headsep=6mm, headheight=4.8mm,
                     top=25mm, bottom=25mm, left=25mm, right=25mm]{geometry}
\RequirePackage{caption} 				% Caption customization
\captionsetup{justification=centerlast,font=small,labelfont=sc,margin=1cm}
\usepackage{hyperref}	%Permite que si hago click en una referencia o link me lleve a ella
\hypersetup{
    colorlinks=true,
    linkcolor=blue,
    filecolor=magenta,      
    urlcolor=blue,
    citecolor=blue,    
}
\usepackage{fancyhdr}
\fancyhead{}
%Acá empezo a escribir en el encabezado
%\lhead{A la izquierda}
%\rhead{\framebox{A la derecha con borde}}
\fancyfoot{}
\rfoot{\thepage}
\pagestyle{fancy}
\renewcommand{\headrulewidth}{0pt}	%Tamaño de la raya que marca en encabezado
\renewcommand{\footrulewidth}{0.5pt}

\usepackage{array}
\newcommand{\PreserveBackslash}[1]{\let\temp=\\#1\let\\=\temp}
\newcolumntype{C}[1]{>{\PreserveBackslash\centering}p{#1}}
\newcolumntype{R}[1]{>{\PreserveBackslash\raggedleft}p{#1}}
\newcolumntype{L}[1]{>{\PreserveBackslash\raggedright}p{#1}}

\usepackage{tikz}
\usetikzlibrary{babel}

\usepackage{amsmath}	%Para manejo dentro de ecuaciones
\usepackage{bm}	%Es el bold map

\usepackage{float}
